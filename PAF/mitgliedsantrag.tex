\documentclass[a4paper,10pt]{article}

\usepackage[hidelinks]{hyperref}
\usepackage{xcolor}
\usepackage{graphicx}
\usepackage[utf8]{inputenc}
\usepackage{lmodern}
\usepackage{textcomp}
\usepackage[skins]{tcolorbox}
\usepackage[left=2cm,top=2cm,right=2cm,bottom=2cm]{geometry}

% from https://tex.stackexchange.com/a/146836
\newtcolorbox{imagetextbox}[3][]{
  before=\par\bigskip\noindent,after=\par\medskip,
  blank,sidebyside,center lower,
  fontlower=\fontsize{25pt}{28pt}\selectfont\bfseries,
  width=\textwidth-#2-#3,
  lefthand width=#2,
  sidebyside gap=#3,#1}

\newcommand{\UnderlinedField}[3][]{\TextField[name=#2,width=#3,bordercolor=black,borderstyle=U,#1]{}}

\pagenumbering{gobble}

\begin{document}

\begin{imagetextbox}[]{8cm}{3mm}
	\includegraphics[width=5cm]{logo_quer}
	\tcblower
	\vspace{0.5cm}
	Mitgliedsantrag
\end{imagetextbox}



\begin{Form} %TODO set [action={<url>}]?

\section*{Ihre Daten}
\begin{tabular}{r r r r}
	\vspace{0.2cm}
    Name*		& \UnderlinedField{lastName}{0.31\textwidth}	& Vorname*		& \UnderlinedField{firstName}{0.31\textwidth} \\
    \vspace{0.2cm}
    Geburtsdatum*	& \UnderlinedField{birthdate}{0.31\textwidth}	& Telefon		& \UnderlinedField{phone}{0.31\textwidth} \\
    \vspace{0.2cm}
    E-Mail		& \multicolumn{3}{l}{\UnderlinedField{email}{0.82\textwidth}} \\
    \vspace{0.2cm}
    Straße*		& \multicolumn{3}{l}{\UnderlinedField{address}{0.82\textwidth}} \\
    \vspace{0.2cm}
    PLZ, Ort*	& \multicolumn{3}{l}{\UnderlinedField{city}{0.82\textwidth}} \\
\end{tabular}
\vspace{0.2cm}\\
Art der Mitgliedschaft*:
	\CheckBox[radio,name=memberType1,bordercolor=black,radiosymbol=\ding{54}]{}{ } Natürliche Person
	\CheckBox[radio,name=memberType2,bordercolor=black,radiosymbol=\ding{54}]{}{ } Familie
	\CheckBox[radio,name=memberType3,bordercolor=black,radiosymbol=\ding{54}]{}{ } Unternehmen, Verein oder Behörde
\\
Der Mitgliedsbeitrag beträgt 24€ / Jahr für natürliche Personen. Familien, Unternehmen, Vereine und Behörden zahlen 48€ / Jahr. Die Familienmitgliedschaft besteht aus einzelnen Accounts für jedes Familienmitglied.\\

\CheckBox[name=constitution,bordercolor=black,checkboxsymbol=\ding{54}]{ } Ich habe die Satzung gelesen und akzeptiere sie* (\href{https://www.bn-paf.de/satzung.pdf}{\texttt{https://www.bn-paf.de/satzung.pdf}})


\section*{E-Mail \& Homepage}

Das Bürgernetz Pfaffenhofen stellt seinen Mitgliedern einen Webserver, Datenbank Server sowie E-Mail Server zur Verfügung. Sofern dies gewünscht ist können dort ein Webseitenverzeichnis sowie E-Mail Adressen mit den Endungen @bn-paf.de und @pfaffenhofen.de angelegt werden.
\\
Sofern noch nicht vergeben möchte ich die folgenden E-Mail Adressen Präfixe: \\
	\UnderlinedField{mailAddresses}{\textwidth}



\section*{Lastschriftmandat}
\fbox
{
	\begin{minipage}{\textwidth}
		\textbf{Zahlungsempfänger (Gläubiger):} \\
		\begin{tabular}{l l}
			Name					& Bürgernetz Landkreis Pfaffenhofen e.V. \\
			Adresse					& Sparkassenplatz 11 \\
			PLZ, Ort				& 85276 Pfaffenhofen \\
			Identifikationsnummer	& DE95BNV00000817521 \\
		\end{tabular}
		\\

		Ich ermächtige das Bürgernetz Landkreis Pfaffenhofen e.V.,
		Zahlungen von meinem Konto mittels Lastschrift einzuziehen.
		Zugleich weise ich mein Kreditinstitut an, die von dem
		Bürgernetz Landkreis Pfaffenhofen e.V. auf mein Konto gezogenen
		Lastschriften einzulösen.
		\\
		
		\textbf{Hinweis: Ich kann innerhalb von acht Wochen, beginnend
		mit dem Belastungsdatum, die Erstattung des belasteten
		Betrages verlangen. Es gelten dabei die mit meinem
		Kreditinstitut vereinbarten Bedingungen.}
		\\
		
		\textbf{Zahlungsart: Wiederkehrende Zahlung}
		\\
		
		\textbf{Zahlungspflichtiger:} \\
		\begin{tabular}{l l}
			IBAN*	& \UnderlinedField{iban}{0.8\textwidth} \\
		\end{tabular}
		\\
		
		\begin{tabular}{l l l}
			Ort*								& Datum*							& Unterschrift* \\
			\UnderlinedField[height=1cm]{signaturePlace}{0.3\textwidth}	& \UnderlinedField[height=1cm]{signatureDate}{0.3\textwidth}	& \UnderlinedField[height=1cm,readonly]{signature}{0.3\textwidth} \\
		\end{tabular}
	\end{minipage}
}

\vspace{0.8cm}
Mit * gekennzeichnete Felder sind Pflichtfelder.

\end{Form}


\end{document}